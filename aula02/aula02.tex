% The preceding line is only needed to identify funding in the first footn%\documentclass[journal, onecolumn, letterpaper]{IEEEtran}
%\documentclass[journal,onecolumn]{IEEEtran}
% \documentclass[conference]{IEEEtran}
\documentclass[a4paper, 12pt, onecolumn,singlespacing]{article}

% The preceding line is only needed to identify funding in the first footnote. If that is unneeded, please comment it out.
\usepackage[level]{fmtcount} % equivalent to \usepackage{nth}
% \include{util}
\usepackage[portuguese, brazil, english]{babel}
\usepackage{multirow}
\usepackage{array} % for defining a new column type
\usepackage{varwidth} %for the varwidth minipage environment
\usepackage[super]{nth}
\usepackage{authblk}
\usepackage{cite}
\usepackage{amsmath,amssymb,amsfonts}
\usepackage{ulem}
\usepackage{graphicx}
% \usepackage{subfig}
\usepackage{textcomp}
\usepackage{xcolor}
\usepackage{mathptmx}
\usepackage[T1]{fontenc}
\usepackage{textcomp}
\usepackage{titlesec}
\usepackage{helvet}
\usepackage{gensymb}
\usepackage{setspace} % espacamento entre linhas
\usepackage{pgfplots}
\usepackage{tikz}
\usepackage{subcaption}
\usepackage{minted}
\usepackage[left=2cm, right=2cm, bottom=2cm, top=2cm]{geometry} 
\usepackage{makecell}
\usepackage{pdfpages}
\usepackage{tabularx}
\usepackage{hyperref}
\usepackage{fancyhdr}
\usepackage{subcaption}
\usepackage{colortbl}
\renewcommand{\headrulewidth}{1pt}
\renewcommand{\footrulewidth}{0.5pt}
\fancyhf{} % limpa os cabecalhos e rodapés
\fancyhead[C]{\textit{INTRODUÇÃO AO APRENDIZADO DE MÁQUINA - EELT7023} } % define o cabeçalho personalizado
\fancyfoot[C]{\textit{AUGUSTO MATHIAS ADAMS}}
\pagestyle{fancy} % sem definir esse comando, o cabeçalho personalizado não é exibido

\hypersetup{
	colorlinks=true,
	linkcolor=blue,
	filecolor=magenta,      
	urlcolor=blue,
	pdftitle={INTRODUÇÃO AO APRENDIZADO DE MÁQUINA - EELT7023}
}
\renewcommand\theadalign{bc}
\renewcommand\theadfont{\bfseries}
\renewcommand\theadgape{\Gape[4pt]}
\renewcommand\cellgape{\Gape[4pt]}

%dashed line
\usepackage{booktabs, makecell}
\renewcommand\theadfont{\bfseries}
\renewcommand\theadgape{}
\usepackage{arydshln}
\setlength\dashlinedash{0.2pt}
\setlength\dashlinegap{1.5pt}
\setlength\arrayrulewidth{0.3pt}

% padrao 1.5 de espacamento entre linhas
\setstretch{1.5}
\makeatletter
\def\@maketitle{%
	\newpage
	\null
	\vskip 2em%
	\begin{center}%
		\let \footnote \thanks
		{\LARGE \@title \par}%
		\vskip 1.5em%
		{\large
			\lineskip .5em%
			\begin{tabular}[t]{c}%
				\@author
			\end{tabular}\par}%
		%\vskip 1em%
		%{\large \@date}%
	\end{center}%
	\par
	\vskip 1.5em}
\makeatother

\definecolor{LightGray}{gray}{0.9}

\title{\normalsize{INTRODUÇÃO AO APRENDIZADO DE MÁQUINA - EELT7023}\\ \huge{\textbf\textit{{AULA 02: APRENDIZADO DE MÁQUINA, UMA INTRODUÇÃO }}\\}}
\author{\small{AUGUSTO MATHIAS ADAMS}}
\setcounter{Maxaffil}{0}
\renewcommand\Affilfont{\itshape\small}

\begin{document}
	% Seleciona o idioma do documento
	\selectlanguage{brazil}
	
	% título
	\maketitle
	
	\section{QUESTÃO 1}
	
	\subsection{Enunciado}
	
		Pense num problema real onde existe a necessidade de desenvolvimento de um algoritmo de Aprendizado de Máquina.
	
	\subsection{Resposta}
	
	Um problema relevante ligado ao mercado financeiro é a previsão do movimento de preços de ações. Este problema envolve o desenvolvimento de um modelo que possa prever o preço futuro de uma ação com base em dados históricos e outros indicadores financeiros.
	
	\textbf{\textit{Problema: }} Prever o movimento (alta, baixa ou estabilidade) dos preços de ações com base em dados históricos, permitindo que investidores identifiquem oportunidades de compra ou venda, visando maximizar lucros ou minimizar perdas.
	
	\section{QUESTÃO 2}
	
	\subsection{Enunciado}
	Trata-se de um problema de aprendizado Supervisionado ou Não Supervisionado?
	
	\subsection{Resposta}
	
	Este é um problema de Aprendizado Supervisionado. Há um histórico de preços de ações disponível, e a tarefa é treinar o modelo para reconhecer padrões nesses dados que possam indicar o comportamento futuro dos preços.
	
	\section{QUESTÃO 3}
	
	\subsection{Enunciado}
	Caso seja um problema de aprendizado Supervisionado, trata-se de um problema de Regressão ou Classificação?
	
	\subsection{Resposta}
		Este problema pode ser abordado tanto como um problema de \textit{Regressão} quanto de \textit{Classificação}, dependendo do objetivo:
		\begin{itemize}
			\item 	\textbf{\textit{Regressão:}} Se o objetivo for estimar o preço da ação em um momento futuro (ex.: o preço no fechamento do próximo dia), trata-se de um problema de regressão.
			\item 	\textbf{\textit{Classificação:}} Se o objetivo for prever a direção do movimento do preço (ex.: se o preço irá subir, descer ou permanecer estável), então é um problema de classificação.
		\end{itemize}
		
	\section{QUESTÃO 4}
	
	\subsection{Enunciado}
		Quais serão os dados que você utilizará para alimentar o seu algoritmo?
	
	\subsection{Resposta}
	
		Os dados utilizados para treinar o modelo podem incluir uma ampla gama de variáveis, tais como:
		\begin{itemize}
			\item \textbf{\textit{Preços históricos:}} Preços de abertura, fechamento, máxima, mínimas.
			\item \textbf{\textit{Indicadores técnicos:}} Médias móveis (simples e exponenciais), Índice de Força Relativa (RSI), Média Móvel Convergente/Divergente (MACD), Bandas de Bollinger, entre outros.
			
		\end{itemize}
	
	Para fins de estudo e demonstração, utilizaremos os dados financeiros do Yahoo Finance para a ação \textit{\textbf{PETR4.SA}}, utilizando o indicador \textbf{\textit{OHLC4}} e o Modelo \textit{\textbf{SGD}} conforme indicado no \href{file://Codigo_Problema_02_aula_02.ipynb}{Notebook}.
	
		
\end{document}